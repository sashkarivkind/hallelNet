%%%%%%%%%%%%%%%%%%%%%%%%%%%%%%%%%%%%%%%%%%%%%%%%%%%%%%%%%%%%
%%  This Beamer template was created by Cameron Bracken.
%%  Anyone can freely use or modify it for any purpose
%%  without attribution.
%%
%%  Last Modified: January 9, 2009
%%

\documentclass[xcolor=x11names,compress]{beamer}
\usepackage[utf8]{inputenc} 
\usepackage[T1]{fontenc}
\usepackage{cmbright}
\usepackage{amsmath}

%% General document %%%%%%%%%%%%%%%%%%%%%%%%%%%%%%%%%%
\usepackage{graphicx}
\usepackage{tikz}

\usepackage[style=authoryear]{biblatex}
%\usetikzlibrary{decorations.fractals}
%%%%%%%%%%%%%%%%%%%%%%%%%%%%%%%%%%%%%%%%%%%%%%%%%%%%%%


%% Beamer Layout %%%%%%%%%%%%%%%%%%%%%%%%%%%%%%%%%%
\useoutertheme[subsection=false,shadow]{miniframes}
\setbeamerfont{title like}{shape=\scshape}
\setbeamerfont{frametitle}{shape=\scshape}

\setbeamercolor*{lower separation line head}{bg=DeepSkyBlue4} 
\setbeamercolor*{normal text}{fg=black,bg=white} 
\setbeamercolor*{alerted text}{fg=red} 
\setbeamercolor*{example text}{fg=black} 
\setbeamercolor*{structure}{fg=black} 

\usenavigationsymbolstemplate{}
\setbeamercolor*{palette tertiary}{fg=black,bg=black!10} 
\setbeamercolor*{palette quaternary}{fg=black,bg=black!10} 

\renewcommand{\(}{\begin{columns}}
\renewcommand{\)}{\end{columns}}
\newcommand{\<}[1]{\begin{column}{#1}}
	\renewcommand{\>}{\end{column}}
%%%%%%%%%%%%%%%%%%%%%%%%%%%%%%%%%%%%%%%%%%%%%%%%%%
\useoutertheme{infolines} % authors, etc.
%%%%%%%%%%%%%%%%%%%%%%%%%%%%%%%%%%%%%%%%%%%%%%%%%%

%\bibliography{presentation.bib}
\bibliography{arivkindNeuroBib}	
\newcommand{\sign}{\mathrm{sign}}
\DeclareGraphicsExtensions{.pdf,.png,.jpg, .eps}

\usepackage{color, colortbl}  
\definecolor{LightCyan}{rgb}{0.88,0.8,1}
\begin{document}
	
	
	\begin{frame}{}
		\title{Simplifying scale-free networks}
		\subtitle[short]{Some analytical insights}
		\author{
			Hallel Schreier and Alexander Rivkind}
		\date{\today}
		\titlepage
	\end{frame}
	
%	\begin{frame}{Table of contents}
%		\tableofcontents
%	\end{frame}
	
	\begin{frame}{when lumping works and when it fails}
	 As lumped subset of nodes grows it develops non-trivial internal dynamics which might be unstable.
		 \begin{figure}
		\centering
		\includegraphics[width=0.7\linewidth]{lumping_vs_stability}
		\caption[Stability of frozen core]{m - size of lumped hub. $P_{pm}$ - probability to converge to the original or to fully inverted frozen core.}
		\label{fig:lumpingvsstability}
		\end{figure}
  	\end{frame}
	
	\begin{frame}[t]{How far can we lump?  - setting hub size}
		1. How strong the lumped hub should be to make a network stable.\\
		2. How many nodes can be lumped w/o compromising network properties. 
	\end{frame}	
	
	\begin{frame}[t]{Fixed points are typical for random neural networks}
		The state $s^*=(1,1,1,1,1)^T$ is a fixed point with probability $2^{-N}$, (happens when $\sum_{j=1}^{N} W_{ij}  >0 \quad \forall i $) and same for any other state $s^*$. \\
		As we assume (and will explain in sequel) that \[P\{s^1\:is\:f.p|s^2:is\:f.p \} \approx P\{s^1\:is\:f.p \}\] for any $s^1\ne\pm s^2$.\\
		Remark: for a more general case and regorous treatment - 
		 \[ \mathbf{E}(\#f.p.)=1.\]. 
		 \cite{mori_PhysRevLett.119.028301}.\\
	\end{frame}
	
	\begin{frame}[t]{if the network was dense and gaussian}
		for any pair of trajectories $x^1$, $x^2$ we would have:
		\[
		\langle
		x^{1}_{t+1},x^{2}_{t+1}
		\rangle
		=g^2
		\langle
		\sign (x^{1}_{t}),\sign(x^{2}_{t})
		\rangle
		\] 
		and thus a mapping between correlation $c^{12}_{t+1}=\langle x^{1}_{t+1},x^{2}_{t+1}\rangle$ and $c^{12}_{t}=\langle x^{1}_{t},x^{2}_{t}\rangle$
	\end{frame}
	
	\begin{frame}[t]{By Gaussian MFT works f.p. are \textit{not} stable}		
	\begin{figure}
		\centering
		\includegraphics[width=0.7\linewidth]{cttp1_u}
		\caption{meanfield estimate of correlation evolution shown for different values of external drive u}
		\label{fig:cttp1u}
	\end{figure}	
	\end{frame}
	
		\begin{frame}[t]{Frozen core is upper bounded}
			
		\end{frame}

	
	\begin{frame}[t]{How MFT breaks in star network}
		1. dynamic wise: $W^TW\ne I$.\\
		2. state wise: %todo $x \not \sim   \mathscr{N}$.
	\end{frame}

	\begin{frame}[t]{Limit cycles are long but not very long}
		If all links were random %todo
	\end{frame}

	\begin{frame}[t]{So frozen cores might  survive}
		
	\end{frame}

	\begin{frame}[t]{Limit cycles are long but not very long}
		
	\end{frame}
	
		\begin{frame}[t]{Work in progress - }
			
		\end{frame}		
	
	
%\bibliographystyle{unsrt}

\end{document}